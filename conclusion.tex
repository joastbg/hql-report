\chapter{Conclusion}

In this project we have created the core instrastructure necessary for 
valuation of financial products using Haskell. 

We have modelled a type class hierarchy
for the financial concepts and products that we require for a complete
library for quantitative finance. This also suggests further development
of \hql.

Further, we have pointed out the myriad of conventions that exist in
the financial domain, most of which pertain to the manipulation of dates.
As a result, we have designed calendar and daycount convention modules
that accounts for the most frequently used conventions.\\

We have studied the different components of debt markets which resulted
in modules for interest rates and term structures.
Moreover, we have developed a functional fixed income module based on
Sinan Gabel's \texttt{DerivativesExpert} library with support for five
different types of bonds. As we have seen, this module uses discounting
to compute the present value of fixed income products.

We have also discussed the constraints in Haskell's type system when
creating instances of the top-level \texttt{Instrument} class.

Finally, we have suggested ways in which the \hql project should be improved,
starting with floating-rate bonds and interest rate derivatives.
