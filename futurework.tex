\chapter{Future Work}

As we set out to design the architecture of \hql\hspace{-0.05cm}, our 
project is naturally far 
from a fully-fledged library, and many more modules must be added. Below we
describe our view of the next steps that should be taken in the development
of \hql.

\section{Observables}

As we saw in the introductory section, the value of financial instruments
all share one dependency: market data. For stock options it is necessary 
to get market prices, and for fixed income we study the term structure of
interest rates to discount our cashflows. In other words, this data is 
external to the models we use.\\

Therefore it is desirable to wrap our computations in a monad that allows us
to structure our program in a pure manner. A \texttt{MonadReader} monad is an obvious
choice to carry an environment throughout the program (such as a term structure or
currently quoted exchange rates). Moreover, we would like \hql to handle exceptional
behaviour in an elegant way. If a user attempts to price a cashflow in the past we 
simply fail the computation using Haskell's \texttt{error}. It would be preferable 
to use the \texttt{ErrorT} monad transformer to structure error handling.

\section{Floating Rates}

First of all we do not have support for the ubiquitous floating-rate bonds, 
where we must rely on implied interest rates as discussed in section \ref{sec:ts}
or Monte Carlo simulations for valuation. Haskell's 
\texttt{State} monad provides an excellent interface for a \texttt{Simulation}
type for stochastic processes, which are typically used in pricing floating
rate bonds that use the \emph{London Interbank Offered Rate} (LIBOR). Such
processes are easily created using a \texttt{State} monad, as in the following
example of Brownian motion process using a Box-Muller transform:

\begin{hscode}
type Simulation = State StdGen -- abstract away random number generator

-- | Generates a random number in the [0,1] range
randomUnit :: Simulation Double
randomUnit = state $ randomR (0,1)

-- | Converts a uniform distributin sample into a
-- normal gaussian distributed sample.
boxMuller :: Simulation Double
boxMuller = do
  u0 <- randomUnit
  return $ transform u0 u0
  where transform x y = sqrt(-2*log x )*cos(2*pi*y)

-- | A standard Brownian motion.
-- | Returns a number of steps according to the Int parameter
sbm :: Int -> Simulation [Double]
sbm n = replicateM n boxMuller
\end{hscode}

Since the rate changes at regular intervals, it is very convenient for us to
model floating rate bonds using a monad. 

\section{Fixed Income Derivatives}

As soon as floating rate bonds are implemented it will be possible to model
basic interest rate derivatives such as swaps. A so-called vanilla swap is an
OTC product between two parties, where each agree to pay the other's debt. While
one party pays a fixed interest rate, the other pays a floating rate. The
motivation behind entering such an agreement is typically that the party paying
a floating rate is averse to the risk of rising interest rates, or market
speculation by the counterpart.\\

Swaps can be modelled by combining a fixed and a floating rate bond in a data
type and defining a \texttt{IRS} (interest rate derivatives) class as in figure
\ref{fig:classhier} for appropriate methods.
Using our existing types:

\begin{hscode}
data IRS where
    IRS :: { leg1 :: Bond b => b,
             leg2 :: Bond b => b } -> IRS
\end{hscode}

This definition subsumes the following swap types:

\begin{center}  
\begin{longtable}{r|c|c|}
Swap Type & Fixed          & Floating \\\hline  
Fixed     & Inter-currency & Vanilla\\\hline
Floating  & Vanilla        & Basis\\\hline
\caption{Rates and basic swap types.}
\end{longtable}
\label{tab:swaps}
\end{center}

Basis swap use different reference rates, e.g. one uses LIBOR while the other
uses the Euro Interbank Offered Rate (EURIBOR)\cite{HULL}.

\section{Mortgage-backed Obligations}

Support for mortgage-backed bonds is also missing in \hql. These also fall
into the fixed income category but are different in the sense that the
invidual who took the loan may pay out the remaining outstanding principal
at any time (a \emph{prepayment}). This could be due to selling a house 
and paying back the loan in full. 
We provide an \texttt{MBO} class so that passthroughs or subprimes can share a 
common interface. However, the data structures and functionality are missing 
and should be  implemented. \hql should also be extended to allow 
prepayment estimates to be computed.

\section{OTC Products}

It would be desirable to implement an embedded domain-specific language (EDSL)
for construction of OTC products. Peyton Jones and Eber present such a contract
language in \cite{composingcontracts}. As OTC products are subject to
counterparty risk, this would be another topic to investigate. Note that government
issued bonds are also affected by counterpart risk since a state may declare
a sovereign default.

\section{Bootstrapping}

So far we have seen how to generate cashflows and how to obtain their 
discounted values. However, this is preconditioned on the fact that we already 
have a term structure available. This begs the question, how do we construct 
such a term structure if all we have is a set of yield curves whose yields 
are inherently dependent on the given instrument we wish to price?\\

Bootstrapping is a method of generating such a term structure from yield 
curves, making it possible to discount any cash flow according to the 
prevailing interest rates. 
This would allow users to import the interest rates that are currently being 
offered in the market from external sources e.g. large financial institutions like
as \href{www.bloomberg.com}{Bloomberg} or \href{www.reuters.com}{Reuters}.
This data could then be used to build a \texttt{TermStructure} type 
that may be used for bond valuation. Note that \hql's discrete and interpolated 
term structures would be easily implemented, but the analytical term structure
would require application of methods such as cubic splines or term structures
by the Nelson-Siegel method. 
An interesting project could set out to investigate the feasibility of adding such 
functionality to \hql.

\section{Dates, Dates, Dates}

Our calendar module needs to be revised as we currently interpolate 
payment dates using the settlement date, maturity and the desired amount of 
settlements per year. However, this is too lax a method, since most standard 
products have set payment patterns. A way of extending this could be to make
the interface stricter so that users must specify a standard, well-defined
payment structure or define their bond as an OTC product.\\

Moreover, \hql currently assumes that all weekdays are valid payments dates. 
This does not cater to local holidays (e.g. \emph{Store Bededag} in Denmark). A 
solution is to compute these holidays using offsets from Easter Sunday, 
alleviating the need to store them statically.\\

