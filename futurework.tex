\chapter{Future Work}

As we set out to design the architecture of HQL, our project is naturally far 
from a fully-fledged library, and many more modules must be added.

First of all we do not have support for the ubiquitous floating-rate bonds (see 
appendix X section something), where we must rely on Monte Carlo simulations 
for valuation.

A natural next step would then be to implement interest rate derivatives such 
as swaps, caps, floors [ref].
There are two obvious ways in which we can extend HQL with interest rate 
derivatives support. For instance, swaps can be modelled by combining a fixed 
and a floating leg. On the other hand, it could be constructed as a list of 
forward-rate agreements (make sure this is explained above, or refer to 
Appendix) between the two parties in question.


Also, our calendar module needs to be revised as we currently interpolate 
payment dates using the settlement date, maturity and the desired amount of 
settlements per year. However, this is too lax a method, since most standard 
products have set payment patterns. Interpolation is bordering on OTC products.
Moreover, HQL currently assumes that all weekdays are valid payments dates. 
This doesn’t cater to local holidays (e.g. Store Bededag in Denmark). A 
solution is to compute these holidays using offsets from Easter Sunday, 
alleviating the need to store them statically.

Another interesting extension is to implement a way of generating yield curves 
from portfolios using bootstrapping or interpolation [ref] (linear equations 
(C. Munk p26-27)). This would essentially allow users to import data from 
external sources such as Bloomberg, Reuters, and build curves in HQL for bond 
valuation.

Support for mortgage-backed bonds is also missing in our library. We provide an 
MBO class so that Passthroughs/MBOs can share a common interface. Prepayment 
estimates must also be computed.

Cubic splines or Nelson-Siegel method for obtain term structure as a function.

Finally, it would be desirable to implement an embedded domain-specific 
language (EDSL) for construction of “over-the-count” (OTC) products.
As OTC products are subject to counterparty risk, this would be another topic 
to investigate.

The class hierarchichy should be extended to define interfaces for derivatives, 
commodities, money markets etc.. Draw diagram?


Alas, our library is missing a key component which is a proper calendar module.


Compute holidays using Easter Sunday

------------------

First of all we do not have support for the floating rate bonds (see appendix X 
section something) due to the lack of of Monte Carlo simulation module. A 
natural next step after floating rate support would then be to implement 
interest rate derivatives such as swaps. There are two obvious ways in which we 
can extend HQL with interest rate derivatives support. For instance, swaps can 
be modelled by combining a fixed and a floating leg, the latter still not yet 
being supported. On the other hand, it could be built as a list of forward-rate 
agreements (make sure this is explained above, or refer to Appendix) between 
the two parties in question.
Alas, our library is also missing a key component which is a proper calendar 
module.
We currently allow the users to supply a variable number of settlements per 
year. This needs to be constrained so that the users either specify a standard 
settlement pattern (missing an example) or all payment dates for OTC products. 
Moreover, HQL currently assumes that all weekdays are valid payments dates. 
This doesn’t cater to local holidays (e.g. January 1 st in Denmark, 4 th of 
July in the United States, etc) which may be computed as offsets from Easter 
Sunday.

Another interesting extension is to implement a way of generating term 
sructures from portfolios using bootstrapping and interpolation 1 . This would 
essentially allow users
to import data from external sources and use HQL to build term structures and 
perform
bond valuation.
Support for mortgage-backed bonds is also missing in our library. We provide an 
MBO
class to be used in the future.
Finally, it would be desirable to implement an embedded domain-specific language
for construction of over-the-counter (OTC) products. As OTC products are 
subject to
counterparty risk, this would be another topic to investigate.

\section{Bootstrapping}

So far we have seen how to generate cashflows and how to obtain their 
discounted values. However, this is preconditioned on the fact that we aldready 
have a term structure available. This begs the question, how do we construct 
such a term structure if all we have is a set of yield curves who’s yields 
are inherently dependent on the given instrument we wish to price?
Bootstrapping is method of generating such a term structure from the yield 
curves, making it possible to discount any cash flow according to the 
prevailing interest rates.
This requires the design of an algorithm
\cite{HULL} provides excellent description of how bootstrapping is performed.
