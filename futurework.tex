\chapter{Future Work}

As we set out to design the architecture of \hql, our project is naturally far 
from a fully-fledged library, and many more modules must be added. Below we
describe the next steps that should be taken in the development of \hql.

\section{Floating rates}

First of all we do not have support for the ubiquitous floating-rate bonds, 
where we must rely on Monte Carlo simulations for valuation. Haskell's 
\texttt{State} monad provides an excellent interface for a \texttt{Simulation}
type for stochastic processes, which are typically used in pricing floating
rate bonds that use LIBOR. We present a small proof of concept of a brownian
motion process using a\href{http://en.wikipedia.org/wiki/Box%E2%80%93Muller_transform}{Box-Muller} transform:

\ab{Find something more suitable than Wikipedia...}

\begin{hscode}
type Simulation = State StdGen -- abstract away random number generator

randomUnit :: Simulation Double
randomUnit = do
               gen <- get
               let (v,gen') = randomR (0::Double, 1::Double) gen
               put gen'
               return v

boxMuller :: Simulation Double
boxMuller = do
  u0 <- randomUnit
  return $ transform u0 u0
  where transform x y = sqrt(-2*log x)*cos(2*pi*y)

-- | A standard Brownian motion
sbm :: Simulation [Double]
sbm = sequence $ repeat boxMuller
\end{hscode}

\section{Fixed Income Derivatives}

A natural next step would then be to implement interest rate derivatives such 
as swaps. There are two obvious ways in which we can extend \hql with interest rate 
derivatives support. For instance, swaps can be modelled by combining a fixed 
and a floating leg. On the other hand, it also could be constructed as a list of 
forward-rate agreements between the two parties in question.\\

\ab{Make sure forward-rate agreements are explained above!}

\section{Mortgage-backed Obligations}

Support for mortgage-backed bonds is also missing in our library. We provide an 
\texttt{MBO} class so that passthroughs or subprimes can share a common interface.
However, the data structures and functionality are missing and should be 
implemented. \hql should also be extended so as to allow prepayment estimates to
be computed.

\section{OTC products}

It would be desirable to implement an embedded domain-specific 
language (EDSL) for construction of OTC products. As OTC products are
subject to counterparty risk, this would be another topic 
to investigate.

\section{Bootstrapping}

So far we have seen how to generate cashflows and how to obtain their 
discounted values. However, this is preconditioned on the fact that we aldready 
have a term structure available. This begs the question, how do we construct 
such a term structure if all we have is a set of yield curves who's yields 
are inherently dependent on the given instrument we wish to price?
Bootstrapping is method of generating such a term structure from the yield 
curves, making it possible to discount any cash flow according to the 
prevailing interest rates.
\cite{HULL} provides excellent description of how bootstrapping is performed.

Regarding term structures, Munk\cite{cmunk} presents several methods of 
approximating
a term structure by means of cubic splines or the Nelson-Siegel method. An 
interesting project could set out to investigate the feasibility of added such 
functionality to \hql.

Another interesting extension is to implement a way of generating yield curves 
from portfolios using bootstrapping or interpolation\cite{cmunk}.
This would allow users to import data from external sources large financial
institutions such as \href{www.bloomberg.com}{Bloomberg} or
\href{www.reuters.com}{Reuters}, and build curves in HQL for bond valuation.\\

\section{Dates, dates, dates}

Our calendar module needs to be revised as we currently interpolate 
payment dates using the settlement date, maturity and the desired amount of 
settlements per year. However, this is too lax a method, since most standard 
products have set payment patterns. A way of extending this could be to make
the interface stricter so that users must specify a standard payment 
structure or define their bond as an OTC product.
\ab{Can we find a few examples? (Sinan?)}

Moreover, \hql currently assumes that all weekdays are valid payments dates. 
This doesn’t cater to local holidays (e.g. \emph{Store Bededag} in Denmark). A 
solution is to compute these holidays using offsets from Easter Sunday, 
alleviating the need to store them statically.\\

