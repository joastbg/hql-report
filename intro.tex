\chapter{Introduction}

Our everyday lives becoming increasingly dependent on IT, and as a result,
software errors are manifold. The financial sector has a particular low tol-
erance, as erroneous software may have dire consequences in the form of
massive monetary loss. Consequently, financial institutions' valuation and risk
management tools must become more sophisticated, putting stress on the quality
of the software.\\

A language that by design allows us to discover a large portion of software
errors in the compilation phase is therefore preferable in the financial domain.\\

We present the the architecture for a library for quantitative finance. We have
implemented it using the pure functional programming language Haskell which
features many attractive properties that steer us toward producing safe and
correct code.\\

The library is intended to be used for the following purposes:\\

\begin{itemize}
\item Numerical valuation of financial contracts with closed-form solutions
\item Simulations and backtesting
\item Explore and develop new contract types from existing ones
\item Research and education
\end{itemize}

By designing and building our software using Haskell, we hope to elim-
inate some of the unnecessary risk in financial modelling and this project
will investigate these possibilities.\\

The project was conducted within the \textsc{Hiperfit} research center at the
University of Copenhagen.\cite{Haskell}
