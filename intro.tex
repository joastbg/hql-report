\chapter{Introduction}

Our everyday lives are becoming increasingly dependent on IT, and as a result,
software errors are manifold. The financial sector has a particular low 
tolerance, as erroneous software may have dire consequences in the form of
massive monetary loss. Consequently, this puts stress on the sophistication and 
quality of valuation and risk management tools employed by financial institutions.\\
A language that by design allows us to discover a large portion of software
errors in the compilation phase is therefore preferable in the financial domain.\\

We present the the architecture for a library for quantitative finance. We have
implemented it using the pure functional programming language Haskell which
features many attractive properties that steer us toward producing safe and
correct code. 

The desired functionality is modelled after Sinan Gabel's 
\texttt{DerivativesExpert}\cite{Mathematica:DerivativesExpert},
a library for \href{http://www.wolfram.com/mathematica/}{Mathematica}
for valuation of financial products such as bonds, interest rate derivatives 
and options. This library formed the basis of our project, and we aim to create
the core Haskell architecture for valuation and re-engineer a subset of DerivativesExpert, 
(mainly fixed income securities) into Haskell.

The library is intended to be used for the following purposes:

\begin{itemize}
\item Numerical valuation of financial contracts with closed-form solutions
\item Simulations and backtesting
\item Explore and develop new contract types from existing ones
\end{itemize}

By designing and building our software using Haskell, we hope to eliminate 
some of the unnecessary risk in financial modelling and this project
will investigate these possibilities.\\

The project was conducted within the \textsc{Hiperfit} research center at the
University of Copenhagen.
