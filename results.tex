\chapter{Results}

\section{Evaluation}

\section{Architecture}

\section{Performance}

\section{Precision and Symbolic Computations}

\chapter{Future Work}

Our project is far from a fully-fledged library, and many more modules must be added.

First of all we do not have support for the ubiquitous floating-rate bonds (see appendix X section something), where we must rely on Monte Carlo simulations for valuation.

A natural next step would then be to implement interest rate derivatives such as swaps, caps, floors [ref].
There are two obvious ways in which we can extend HQL with interest rate derivatives support. For instance, swaps can be modelled by combining a fixed and a floating leg, the latter still not yet being supported. On the other hand, it could be built as a list of forward-rate agreements (make sure this is explained above, or refer to Appendix) between the two parties in question.

Another interesting extension is to implement a way of generating yield curves from portfolios using bootstrapping or interpolation [ref] (linear equations (C. Munk p26-27)). This would essentially allow users to import data from external sources such as Bloomberg, Reuters, and build curves in HQL for bond valuation.

Support for mortgage-backed bonds is also missing in our library. We provide an MBO class so that Passthroughs/MBOs can share a common interface. Prepayment estimates must also be computed.

Cubic splines or Nelson-Siegel method for obtain term structure as a function.

Finally, it would be desirable to implement an embedded domain-specific language (EDSL) for construction of over-the-counter (OTC) products.
As OTC products are subject to counterparty risk, this would be another topic to investigate.

The class hierarchichy should be extended to define interfaces for derivatives, commodities, money markets etc.. Draw diagram?

Alas, our library is missing a key component which is a proper calendar module.

Compute holidays using Easter Sunday

\chapter{Conclusion}
