\chapter{Background}

A financial security is a legal claim on a firm's asset or income 
that is traded in an organized market\cite{alexander2008market}.
A portfolio is  a number of positions in such securities, equities, bonds
and other types of financial instruments. 
Such instruments are typically traded on an exchange such as 
\href{http://www.nasdaqomxnordic.com/}{NASDAQ OMX Nordic}, but so-called
\emph{over-the-counter} (OTC) products also exist where two parties enter
a unilateral, legally binding, trade.\\
The value of such a collection can be computed to reflect the theoretical 
value they would have if they existed today to allow investors to assess
whether resources could yield higher returns elsewhere.\\
Financial instruments are priced differently, some using closed-form functions 
and others relying on stochastic simulation. The former plays a large part in
our project, while the latter is scoped out.\\

Such analyses involve an understanding of the industries in question,
economics, mathematics and computer science. For instance,
domain-specific knowledge in the energy 
industry is a prerequisite for fruitful investing in oil. Similarly, economists 
must apply their knowledge to best assess market trends and constantly reevaluate
the forecasts they base their trading on.
Further, people known as \emph{quants} combine mathematics and computer science
in creating and implementing models for risk management, asset pricing or 
algorithmic trading.\\
In this project we seek to coalesce mathematical finance and software engineering
into a software tool set employed by \emph{quants}.\\

There exist two pivotal components that are key to the pricing of assets in the
fixed income category. The first is discounting which is a method of scaling
future cashflow so as to represent its  theoretical value it would have had
it existed today.
As the prevailing interest rate is observable, the time value of money enables 
the investor to assess whether an investment is up to par. A promise of a 
payment in the future therefore has to be proportional to what the investor 
otherwise would have been able to receive in the market.\\

Secondly is credit risk is a method of evaluating the creditworthiness of a 
debitor, i.e. quantifying how likely it is that the counterpart will pay back 
its debt. This  is not considered for exchange-traded products, as losses 
caused by defaults 
are covered by the members of the exchange\footnote{If a large exchange 
defaults the world economy would be in utter turmoil.}. However, for OTC
products, the evaluation of credit risk is critical as the consequences of a 
default of the counterparty lies solely with the opposing party. Since the 
financial crisis of 2008 this has become increasingly important for financial 
institutions, and \href{http://www.lchclearnet.com/}{clearing houses} have
been established to mitigate the 
domino effect of collapses of large financial institutions.
The time value of money and credit risk are two key components allowing 
investors to gauge whether an investment will be in their favour.\\

\section{Related work}

In terms of free software, the amount of quantitative finance libraries is 
limited.
The most prominent is Quantlib, an open-source C++ library\cite{Ame2003}
supporting a wide array of financial products ranging from bonds to more 
exotic derivatives.\\

Hquantlib\cite{hquantlib} is an attempt at a Haskell port of Quantlib
and is under 
development.\\

Peyton Jones and Eber\cite{composingcontracts} developed a combinator library 
allowing users to define virtually any financial contract by means of a
computational denotational semantics.
\ab{Should this be more explicit?}

Sinan Gabel developed DerivativesExpert\cite{Mathematica:DerivativesExpert}
, an extension for \href{http://www.wolfram.com/mathematica/}{Mathematica}
for valuation of financial products such as bonds, interest rate derivatives 
and options. Our project aims to create the core Haskell architecture for 
valuation of such products and re-engineer a subset of DerivativesExpert, 
(mainly fixed income securities) into Haskell.
