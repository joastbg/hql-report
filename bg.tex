\chapter{Background}

A financial security is a tradable legal claim on a firm's asset or income 
that is traded in an organized market [Alexander, Market Risk Analysis].
A portfolio is  a number of positions in such securities such as long or short 
positions in equities, bonds and other types of financial instruments. The 
present value of such a collection can be computed through discounting and 
credit ratings to  reflect the theoretical value they would have if they 
existed today. This allows investors to assess whether resources could yield 
higher returns elsewhere.\\

The time value of money is a concept that allows investors to assess whether an 
investment is up to par. In essence, it tells us how high a return an 
investment should yield as a function of the time the resources are tied up.
Discounting is a method of scaling a future cashflow so as to represent its 
theoretical value it would have had it existed today.
As the prevailing interest rate is observable, the time value of money enables 
the investor to assess whether an investment is up to par. A promise of a 
payment in the future therefore has to be proportional to what the investor 
otherwise would have been able to receive in the market.\\

Credit risk is a method of evaluating the creditworthiness of a debitor, i.e. 
quantifying how likely it is that the counterpart will pay back its debt. This 
is not considered for exchange-traded products, as losses caused by defaults 
are covered by the members of the exchange\footnote{If the exchange itself 
defaults, everyone is royally screwed anyway}. However, for so-called 
\emph{over-the-counter} products (OTC), where two parties enter an agreement, 
the evaluation of credit risk is critical as the consequences of a default of 
the counterparty lies solely with the opposing party. Since the financial 
crisis of 2008 this has become increasingly important for financial 
institutions [ref], and clearing houses have been established to mitigate the 
domino effect of collapses of large financial institutions.
The time value of money and credit risk are two key components allowing 
investors to gauge whether an investment will be in their favour.\\

Financial instruments are priced differently, some using closed-form functions 
and others relying on stochastic simulation, the latter being scoped out of 
this project.\\

We now briefly describe the financial products in the category of fixed income, 
and we refer appendix A for a more in-depth treatment.\\

- Explain basics of valuation
- Why this is important
	- Risky business
	- Massive loss


\section{Related work}

In terms of free software, the amount of quantitative finance libraries is 
limited.
The most prominent is Quantlib, an open-source C++ library[ref] supporting a 
wide array of financial products ranging from bonds to more exotic derivatives.\\

Hquantlib[link] is an attempt at a Haskell port of Quantlib and is under 
development.\\

Peyton Jones and Eber [ref] developed a combinator library allowing users to 
define virtually any financial contract by means of a computational 
denotational semantics (this is wrong, reread article + denotational semanctics 
+ Tacking the Awkward Squad + Composing Contract summary/references).\\

Sinan Gabel developed DerivativesExpert[ref], an extension for Mathematica[ref] 
for valuation of financial products such as bonds, interest rate derivatives 
and options. Our project aims to create the core Haskell architecture for 
valuation of such products and re-engineer a subset of DerivativesExpert, 
mainly fixed income securities) into Haskell.
