\chapter{Evaluation}

We now assess the quality of our library and address the limitations that we
have encountered throughout the course of the project.

\section{Deliverables}

We have successfully designed and implemented the foundation for a valuation 
library in Haskell. We have performed extensive convention mining in the domain
of finance and implemented the scaffolding required for valuation, 
namely a calendar module, day-count conventions, currency support, interest 
rates and term structures. Moreover, we have also implemented a fixed income 
module which we have used as an example throughout this report.\\
Finally we have created documentation for the modules related to fixed 
income.

\section{Language limitations}

First we point out the limitations in terms of the choice of language.
Haskell's type system poses a challenge when we introduce
\emph{impure} code - namely the \texttt{IO} monad.

For instance, to model inter-currency swaps we need to retrieve the exchange 
rate at the exact moment when the transaction is done. Retrieving this 
information is impure, as it is an external to the Haskell runtime. In Java, a 
programmer would wrap such a query in a try-catch and not think twice about it,
but we do not have this luxury in Haskell due to the stickiness of the 
\texttt{IO} monad. One option is to use \texttt{unsafePerformIO} - but this is 
infeasible as it hides impurity into our code.
Instead we can wrap our models in a monad that will perform the necessary 
computations when given an environment. This solution may also be used to 
retrieve market data and bootstrap a term structure internally.\\

Moreover, \hql should ideally be extended to use a suitable library for
IEEE floating point arithmetic as there are a number of issues regarding
comparisons of doubles described in \cite{blogpost}.\\

Finally, we have encountered a minor limitation in Haskell's class system. This 
becomes apparent when creating instances of various instruments. We return to 
our serial example and show how it is made an instance of \texttt{Instrument}:

\begin{hscode}
instance Instrument FixedAmortizedBond where
  type PricingEngine = TermStructure
  ...
\end{hscode}

Here \texttt{TermStructure} is the Haskell data type described in section 
\ref{sec:ts}. We would rather be able to do the following:
 
\begin{hscode}
instance Instrument FixedAmortizedBond where
  type TermStructure t => PricingEngine = t
  ...
\end{hscode}

The distinction lies in the fact that in the second example, the term structure
is a class, meaning we may use the interface it defines. It would make for a much
cleaner solution.

For instance, if we were to add a new term structure we could simply make it
an instance of the \texttt{TermStructure} class. Instead, we now have to add it as a 
constructor in the \texttt{TermStructure} data type and extend or create all of
the functionality around it.\\

Regarding type safety, a class is also superior because we could potentially
create a hierarchy of pricing engines and use constraints in our instances as 
above. It would be a more organized design of the pricing engine architecture.

%\section{Limitations}
%
%We now point out the limitations that the Haskell language and our architecture 
%have imposed.

%Moreover, Haskell’s type system has not provided us with an elegant solution 
%for representation of currencies.

%GADT/Type families etc are not suited for representing Currency type 
%with the properties we wanted (+,*,/, etc)

%Due to the myriad of conventions in finance, it is desirable to have an 
%interface that allows the user to only specify a subset of the required 
%information for a financial product, and have the the internals default to 
%certain value. For instance in the MBO market, the day count convention is 
%typically 30/360 (verify and reference), so it would be nice to have 

%In C++, this is achieved through function overloading, which is arguably a more 
%elegant interface to exporting Haskell functions that return data types with 
%pre-set fields.
