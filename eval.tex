\chapter{Evaluation}

We now assess the quality of our library and address the limitations that we
have encountered throughout the course of the project.

\section{Architecture}

We have successfully designed and implemented the foundation for a valuation 
library in Haskell. We have performed extensive convention mining in the domain
of finance and implemented the scaffolding required for valuation, 
namely a calendar module, day-count conventions, currency support, interest 
rates and term structures. Moreover, we have also implemented a fixed income 
module which we have used as an example throughout this report.\\

Finally we have created documentation for the modules related to fixed 
income\cite{hqldoc}.

\section{Language limitations}

First we point out the limitations in terms of the choice of language.
Haskell's type system is a double-edged sword which becomes explitic as soon 
as we need to do introduce \emph{impure} code - namely the \texttt{IO} monad.

For instance, to model inter-currency swaps we need to retrieve the exchange 
rate at the exact moment when the transaction is done. Retrieving this 
information is impure, as it is an external to the Haskell runtime. In Java, a 
programmer would wrap such a query in a try-catch and not think twice about it, 
but we do not have this luxury in Haskell due to the stickiness of the 
\texttt{IO} monad. One option is to use \texttt{unsafePerformIO} - but this is 
infeasible as it would introduce impurity into our code.
Section \ Instead we can wrap our models in a monad that will perform the necessary 
computations when given an environment. This solution may also be used to 
retrieve market data and bootstrap a termstructure internally.

Also, we currently use Haskell's \texttt{Double} datatype to 
Ideally HQL should be extended to use the \href{https://hackage.haskell.org/package/data-binary-ieee754}{\texttt{IEEE754}} package for floating 
points, as it correctly addresses the problems regard comparisons of 
doubles described in \cite{blogpost}.

\section{Limitations}

We now point out the limitations that the Haskell language and our architecture 
have imposed.

%Moreover, Haskell’s type system has not provided us with an elegant solution 
%for representation of currencies.

%GADT/Type families etc are not suited for representing Currency type 
%with the properties we wanted (+,*,/, etc)

%Due to the myriad of conventions in finance, it is desirable to have an 
%interface that allows the user to only specify a subset of the required 
%information for a financial product, and have the the internals default to 
%certain value. For instance in the MBO market, the day count convention is 
%typically 30/360 (verify and reference), so it would be nice to have 

%In C++, this is achieved through function overloading, which is arguably a more 
%elegant interface to exporting Haskell functions that return data types with 
%pre-set fields.
