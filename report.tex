\documentclass[a4paper,11pt]{report}

\def\Author{Andreas Bock, bock@andreasbock.dk\\
Johan Astborg, joastbg@gmail.com\\\\
Supervisors:\\
Jost Berthold, jb.diku@gmail.com\\
Sinan Gabel, sinan.gabel@gmail.com
}
\def\Title{\bf HQL - \textsc{Hiperfit} Quant Library\\ {\Large Project Report}}

%
%--------------------   start of the 'preamble'
%
\usepackage{graphicx,amssymb,amstext,amsmath,graphics,epsfig,color}
\usepackage{fancyhdr}
\usepackage{algorithm}
\usepackage{algorithmic}
\usepackage{lmodern,inconsolata}
\usepackage{array, xcolor, lipsum, bibentry, fancyhdr}
\usepackage[absolute]{textpos}
\usepackage[top=25mm, bottom=25mm, left=22mm, right=22mm]{geometry} %Layout of page
\usepackage{lastpage} % number of last page 
%
%%    homebrew commands -- to save typing
\newcommand\etc{\textsl{etc}}
\newcommand\eg{\textsl{eg.}\ }
\newcommand\etal{\textsl{et al.}}
\newcommand\Quote[1]{\lq\textsl{#1}\rq}
\newcommand\fr[2]{{\textstyle\frac{#1}{#2}}}
\newcommand\miktex{\textsl{MikTeX}}
\newcommand\comp{\textsl{The Companion}}
\newcommand\nss{\textsl{Not so Short}}

\usepackage[T1]{fontenc} % font
\setlength{\parindent}{0in}
\definecolor{lightgray}{rgb}{0.9,0.9,0.9}

\newenvironment{filecode}[1][]
{\minipage{\linewidth}
\lstset{basicstyle=\ttfamily\footnotesize,frame=single,
numberstyle=\small\color{black},keywordstyle=\color{black},commentstyle=\color{black},
stringstyle=\color{black},tabsize=2,backgroundcolor=\color{lightgray},language=Haskell,#1}}
{\endminipage}
\renewcommand*\rmdefault{ppl}

\pagestyle{fancy}
\fancyhf{} 
 
\lhead{\uppercase{Hiperfit Quant Library}}
\rhead{\nouppercase{\rightmark}}
 
\cfoot{\thepage\ / \phantomsection\pageref*{LastPage}}
 
\lfoot{
\begin{textblock*}{100mm}(30mm, 280mm )
\end{textblock*}
}

%
%---------------------   end of the 'preamble'
%
\newcommand{\HRule}{\rule{\linewidth}{0.5mm}}

\pagestyle{fancy}               % Fräcka sidhuvuden
\addtolength{\headwidth}{2cm}   % Sidhuvd bredare än texten.
\renewcommand{\headrulewidth}{0.4pt} % Linje i sidhuvud är 0.4 punkter
%\renewcommand{\footrulewidth}{0.4pt} % Linje i sidfot är 2 punkter

% Följande kommandon definerar vad som ska finnas i sidhuvud och
% sidfot. Om man skriver dubbelsidiga dokument anger man två alternativ
% med komma mellan. Den första gäller då för udda sidor och den andra
% för jämna sidor. Bokstäverna ska tolkas som:
% L = left, C = center, R = right,
% E = even (jämna sidor), O = odd (udda sidor)
% E och O fyller ingen funktion om man inte har optionen twopage definierad
\fancyhead[R]{\bf{\nouppercase{\leftmark}}}	% Vänstertext i sidhuvud
\fancyhead[L]{\nouppercase{\rightmark}}	% Högertext i sidhuvud
%\fancyfoot[C]{}	% Mittentext i sidfot
%\fancyfoot[LO,RE]{}		% Vänster udda, höger jämna sidor
%\fancyfoot[LE,RO]{}	% Vänster jämna, höger udda sidor


\begin{document}
%-----------------------------------------------------------
\begin{titlepage}

\textsc{\LARGE }\\[1.5cm]
\textsc{\Large }\\[0.5cm]
\textsc{\large }\\[0.5cm]
 
% Title
\begin{center}
\HRule \\[0.5cm]
\huge \bfseries \Title\\[0.5cm]
\HRule \\[0.5cm]

% Author
\Large
\emph{Author:}\\
\textsc{Andreas Bock \\ Johan Astborg }\\[3cm]


\date{\today}



% Bottom of the page
{\large January 25, 2011}\\[4cm]
%\includegraphics{Logo}\\[1cm] % Department/University logo
 
\vfill
\end{center}

\end{titlepage}
%-----------------------------------------------------------
\begin{abstract}\centering

The building of a multithreaded, high performance ray tracer supporting
direct and global illumination, shading and meshes.
\end{abstract}
%-----------------------------------------------------------
\tableofcontents
%-----------------------------------------------------------
\listoftables
\addcontentsline{toc}{chapter}{List of Figures}
\listoffigures

\chapter{Introduction}

Our lives becoming increasingly dependent on IT, and as a result, software errors are manifold. The financial sector has a particular low tolerance, as erroneous software may have dire consequences in the form of massive monetary loss.
The financial crisis of 2008 also caused legislators to take a conservative
stance on risk (cite), as the collapse on Lehman Brothers reverberated throughout the world's economies.
Consequently, financial institutions' risk management tools must become more sophisticated, putting stress on the quality of the software.

We present a prototype Haskell library for valuation of financial products.

The project was conducted within the \textsc{Hiperfit} research center at the
University of Copenhagen.

Lorem ipsum \gls{bar} \gls{baz} and \gls{foobar}.

\cite{Borman03raytracingand}
\chapter{Fixed Income}
 \begin{fquote}[John von Neumann][1903-1957]If people do not believe that mathematics is simple, it is only because they do not realize how complicated life is.
 \end{fquote}

This document will describe in detail the building blocks of a modern high performance ray tracer.
Feature list:
\begin{itemize}
\item Choice of Language, C++
\item Basic Whitted ray-tracing
\item Kd-Tree acceleration structure
\end{itemize}
Cras nisi neque, pharetra ac cursus nec, vestibulum sit amet erat. Vivamus eget viverra elit. Sed vehicula augue sit amet nibh convallis volutpat. Sed feugiat posuere nunc a auctor. Nam turpis erat, ultrices sed varius in, tempus nec enim. Donec hendrerit dignissim libero, non lacinia odio congue non. Nulla eu velit urna, ut accumsan nibh. Fusce ligula massa, volutpat ut blandit at, dignissim sed orci. Nulla sed mauris lorem. Mauris nec turpis purus, sed sollicitudin massa. Sed ipsum purus, vestibulum et viverra et, tristique at leo. Cum sociis natoque penatibus et magnis dis parturient montes, nascetur ridiculus mus. Sed gravida, odio a rutrum posuere, diam erat fermentum arcu, sit amet blandit orci metus ac mauris.

\chapter{Fixed Income Pricing}
This document will describe in detail the building blocks of a modern high performance ray tracer.
Feature list:
\begin{itemize}
\item Choice of Language, C++
\item Basic Whitted ray-tracing
\item Kd-Tree acceleration structure
\end{itemize}
Cras nisi neque, pharetra ac cursus nec, vestibulum sit amet erat. Vivamus eget viverra elit. Sed vehicula augue sit amet nibh convallis volutpat. Sed feugiat posuere nunc a auctor. Nam turpis erat, ultrices sed varius in, tempus nec enim. Donec hendrerit dignissim libero, non lacinia odio congue non. Nulla eu velit urna, ut accumsan nibh. Fusce ligula massa, volutpat ut blandit at, dignissim sed orci. Nulla sed mauris lorem. Mauris nec turpis purus, sed sollicitudin massa. Sed ipsum purus, vestibulum et viverra et, tristique at leo. Cum sociis natoque penatibus et magnis dis parturient montes, nascetur ridiculus mus. Sed gravida, odio a rutrum posuere, diam erat fermentum arcu, sit amet blandit orci metus ac mauris.

\chapter{Results}

\section{Evaluation}

\section{Architecture}

\section{Performance}

\section{Precision and Symbolic Computations}

\chapter{Future Work}\label{chap:fw}

Our project is far from a fully-fledged library, and many more modules must be added.

First of all we do not have support for the floating rate bonds
(see appendix X section something) due to the lack of of Monte Carlo
simulation module. A natural next step after floating rate support
would then be to implement interest rate derivatives such as swaps.

There are two obvious ways in which we can extend HQL with interest rate
derivatives support. For instance, swaps can be modelled by combining a
fixed and a floating leg, the latter still not yet being supported. On the
other hand, it could be built as a list of forward-rate agreements (make sure
this is explained above, or refer to Appendix) between the two parties in
question.\\

Alas, our library is also missing a key component which is a proper calendar
module. We currently allow the users to supply a variable number of
settlements per year. This needs to be constrained so that the users either
specify a standard settlement pattern (missing an example) or \emph{all}
payment dates for OTC products. Moreover, HQL currently assumes that all
weekdays are valid payments dates. This doesn't cater to local holidays
(e.g. January $1^{\text{st}}$ in Denmark, $4^{\text{th}}$ of July in the
United States, etc) which may be computed as offsets from Easter Sunday.\\

Another interesting extension is to implement a way of generating term
sructures from portfolios using bootstrapping and interpolation\footnote{need
a reference to Björk/Munk here}. This would essentially allow users to import
data from external sources and use HQL to build term structures and perform
bond valuation.\\

Support for mortgage-backed bonds is also missing in our library. We provide
an MBO class to be used in the future.\\
%Prepayment estimates must also be computed.

%Cubic splines or Nelson-Siegel method for obtain term structure as a function.

Finally, it would be desirable to implement an embedded domain-specific language
for construction of over-the-counter (OTC) products.
As OTC products are subject to counterparty risk, this would be another topic
to investigate.

%The class hierarchichy should be extended to define interfaces for derivatives,
%commodities, money markets etc.. Draw diagram?

\chapter{Conclusion}


%-----------------------------------------------------------
\addcontentsline{toc}{chapter}{\numberline{}Bibliography}
\bibliographystyle{plain}
\bibliography{biblio}
%-----------------------------------------------------------
\appendix
\chapter{Contributions}

\newcounter{treeline}

\newcommand{\treeroot}[1]{% Title
\node[above] at (0,0) {#1};%
\setcounter{treeline}{0}
}

\newcommand{\treeentry}[2]{% Title, Level
\draw[->] (#2-1,-\value{treeline}/2) -- (#2-1,-\value{treeline}/2-0.5) -- (#2+0.5,-\value{treeline}/2-0.5) node[right] {#1};
\stepcounter{treeline}
}

\newcommand{\altentry}[2]{% Title, Level
\draw[->] (#2-1,-\value{treeline}/2) -- (#2-1,-\value{treeline}/2-0.5) -- (#2+0.5,-\value{treeline}/2-0.5) node[right] {#1};
\foreach \x in {1,...,#2}
{   \draw (\x-1,-\value{treeline}/2) -- (\x-1,-\value{treeline}/2-0.5);
}
\stepcounter{treeline}
}

\textcolor{blue}{Andreas Bock}\\
\textcolor{green}{Johan Astborg}\\
\textcolor{red}{Both}\\

\begin{tikzpicture}
\tt
\treeroot{src}
\treeentry{Instruments}{1}
\treeentry{\color{blue} Instrument.hs}{2}
\treeentry{FixedIncome}{2}
\treeentry{Bonds}{3}
\treeentry{\color{blue} Bonds.hs}{4}
\treeentry{Utils}{2}
\treeentry{\color{green} InterestRate.hs}{2}
\treeentry{\color{red} TermStructure.hs}{2}
\treeentry{Derivatives}{2}
\treeentry{\color{blue} Derivatives.hs}{3}
\treeentry{Equity}{2}
\treeentry{OTC}{2}

\treeentry{Pricing}{1}
\treeentry{Utils}{1}
\treeentry{\color{blue} Calendar.hs}{2}
\treeentry{\color{red} Currency.hs}{2}
\treeentry{\color{blue} DayCount.hs}{2}
\treeentry{Graphics}{2}
\treeentry{\color{green} Visualize.hs}{3}
\end{tikzpicture}

%-----------------------------------------------------------
\end{document}
