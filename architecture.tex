\chapter{Architecture}

\section{Language}

We have chosen Haskell as modeling language for several reasons. Firstly,
a pure functional language such as Haskell lends itself well to the domain of
finance due to its modularity\cite{hughes:matters-cj} and laziness\cite{composingcontracts}[Section 5.3].
Moreover, the mathematical nature of finance makes functional languages suitable
due its inherent mathematical foundation.

Secondly, as Haskell is a statically typed language it has the advantage of
discovering as many errors at compile time as possible, allowing for safer
 code. Additionally, Haskell enables us to overload functionality in our
programs based on our types. This is a very desirable feature, as we want
 our financial products to have the same API (e.g. a function called
\texttt{presentValue} should return the present value regardless of the
product type).

\section{Design principles}

We set out to provide an intuitive user-friendly interface whilst keeping the
internal valuation engine as simple as possible to cater for transparency of
the mathematics of finance.
We now go through the design of the HQL and how we have utilized the features
of Haskell.

\subsection{Frontend}

We use Haskell data types to represent our bonds so that we may make them
instances of type classes. We refer to figure \emph{X} shows the (inheritance)
relations as this is best explained pictorially.\ab{Produce a figure when we
are done with the interface. This section should be straight-forward.}

\subsection{Backend}

% List representation
As mentioned in chapter \ref{chap:fi}, a bond can be represented as a (possibly
infinite) list of zero coupon bonds (\texttt{(Date, Cash)} tuples).
This is precisely the architecture we use to represent the bonds internally, as
doing so exploits a number of properties of Haskell.\\

Lists are excellent data structures as operations on them (maps and
folds, for instance) are typically embarassingly parallel. Moreover,
Haskell's laziness combats the potentially large performance cost involved in
evaluating a list of payments for a 30 year bond, for instance. In addition,
the laziness of Haskell also makes it possible to represent a consol bond as
an infinite list.\\

% List in valuation
Another benefit in using a list representation is in valuation. As explained
in Appendix \emph{X}, discounting requires us knowledge of the prevailing zero
rates current observable in the market. Given a complete term
structure, bond valuation becomes a simple \texttt{zipWith (*)} between the list
of cash amounts and the list of discount factors obtained through the dates of
the payments. Furthermore, as a term structure returns an interest rate given
a time to maturity, we can completely avoid the use dates and instead use
offsets from the present.\\

The fact that all bonds are valued in the same way results in less boilerplate
code in the internals, but requires a complete term structure for valuation.
Section \ref{chap:fw} discusses this decision in detail.

\section{Limitations}

Due to the myriad of conventions in finance, it is desirable to have an
interface that allows the user to only specify a subset of the required
information for a financial product, and have the the internals default
to certain value. For instance in the MBO market, the day count convention
is typically $\frac{30}{360}$ (verify and reference), so it would be 

In C++, this is achieved through function overloading, which is arguably a
more elegant interface to exporting Haskell functions that return data types
with pre-set fields.

The stickiness of the IO monad poses a problem for currency exchanges.
Assuming foreign exchange rates can be retrieved from institutions such as
\href{http://www.bloomberg.com/markets/currencies/}{Bloomberg}, we would 
have to contaminate the purity of our software with IO. Ideally, we want to
imbue the internals with implicit currency conversion to accomodate for
intercurrency swaps. \texttt{unsafePerformIO} could be used to fetch the foreign
exchange rates, but this introduces more problems e.g. what happens if the
server we are querying is not live?
In Scala, we would be able to use the \emph{implicits} construct to perform
the conversions.

A \texttt{State}) monad could be a way of solving the problem - we structure
our program to take a set of recent exchange rates (and other "online"
information essential to pricing) and performs the computations.

